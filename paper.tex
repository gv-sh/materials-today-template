% Template for the Materials Today journal
% Version   : 1.0, 30-10-22
% Author    : gv-sh

\documentclass[10pt]{article}
\usepackage[a4paper]{geometry}
\usepackage[utf8]{inputenc}
\usepackage{amsmath}
\usepackage{amssymb}
\usepackage{blindtext}
\usepackage{times}
\usepackage{authblk}
\usepackage{graphicx}
\usepackage{hyperref}

\usepackage{titlesec}
\titleformat{\section}{\bfseries}{\thesection}{10pt}{}
\titleformat{\subsection}{\itshape}{\thesubsection}{10pt}{}
\titleformat*{\paragraph}{\itshape}

\usepackage[singlelinecheck=false]{caption} % To anchor captions to left

\title{Name of the Proceeding}
\author[1]{Alice Smith}
\author[2]{Bob Jones}
\affil[1]{Department of Mathematics, University X}
\affil[2]{Department of Biology, University Y}

\date{}

\begin{document}
\maketitle

% Abstract
\begin{abstract}
\blindtext

\paragraph*{Keywords:} Type your keywords here, separated by semicolons [maximum of 6]
\end{abstract}

\hrule

% Main content
\section{Introduction} Here introduce the paper, and put a nomenclature if necessary, in a box with the same font size as the rest of the paper. The paragraphs continue from here and are only separated by headings, subheadings, images and formulae. The section headings are arranged by numbers, bold and 10 pt. Here follows further instructions for authors.


\subsection{Structure}
\indent Files must be in MS Word only. Figures should be embedded and high resolution files supplied separately.

Please make sure that you use as much as possible normal fonts in your documents. Special fonts, such as fonts used in the Far East (Japanese, Chinese, Korean, etc.) may cause problems during processing. To avoid unnecessary errors you are strongly advised to use the ‘spellchecker’ function of MS Word. Follow this order when typing manuscripts: Title, Authors, Affiliations, Abstract, Keywords, Main text (including figures and tables), 

Acknowledgements, References, Appendix. Collate acknowledgements in a separate section at the end of the article and do not include them on the title page, as a footnote to the title or otherwise.

Bulleted lists may be included and should look like this:
\begin{itemize}
    \item First point
    \item Second point 
    \item And so on
\end{itemize}

Please do not alter the formatting and style layouts which have been set up in this template document. Do not number pages on the front, as page numbers will be added separately for the preprints and the Proceedings. Leave a line clear between paragraphs. 


\subsection{Tables}
All tables should be numbered with Arabic numerals. Every table should have a caption. Headings should be placed above tables, left justified. Only horizontal lines should be used within a table, to distinguish the column headings from the body of the table, and immediately above and below the table. Tables must be embedded into the text and not supplied separately. Below is an example which the authors may find useful.

\begin{table}[h]
\caption{Table caption.}
\centering
\begin{tabular}{lcc}
\hline
An example of a column heading & Column A($t$) & Column B($t$) \\ \hline
Add an entry & 1 & 2 \\ 
Add another entry & 3 & 4 \\
Add a third entry & 5 & 6 \\ \hline
\end{tabular}
\end{table}


\subsection{Construction of References}
References must be listed at the end of the paper. Do not begin them on a new page unless this is absolutely necessary. Authors should ensure that every reference in the text appears in the list of references and vice versa. Indicate references by [1] or [2,3] in the text. 

Some examples of how your references should be listed are given at the end of this template in the `References' section, which will allow you to assemble your reference list according to the correct format and font size.


\subsection{Section headings}
Section headings should be left justified, bold, with the first letter capitalized and numbered consecutively, starting with the Introduction. Sub-section headings should be in capital and lower-case italic letters, numbered 1.1, 1.2, etc, and left justified, with second and subsequent lines indented. All headings should have a minimum of three text lines after them before a page or column break. Ensure the text area is not blank except for the last page.


\subsection{General guidelines for the preparation of your text}
Avoid hyphenation at the end of a line. Symbols denoting vectors and matrices should be indicated in bold type. Scalar variable names should normally be expressed using italics. Weights and measures should be expressed in SI units. All non-standard abbreviations or symbols must be defined when first mentioned, or a glossary provided.


\subsection{Footnotes}
Footnotes should be avoided if possible. Necessary footnotes should be denoted in the text by consecutive superscript letters. The footnotes should be typed single spaced, and in smaller type size (8 pt), at the foot of the page in which they are mentioned, and separated from the main text by a one line space extending at the foot of the column. The Els-footnote style is available in the MS Word for the text of the footnote.

Please do not change the margins of the template as this can result in the footnote falling outside printing range.

\subsection{Illustrations}
All figures should be numbered with Arabic numerals (1,2,3,….). Every figure should have a caption. All photographs, schemas, graphs and diagrams are to be referred to as figures. Line drawings should be good quality scans or true electronic output. Low-quality scans are not acceptable. Figures must be embedded into the text, but should also be uploaded separately. Lettering and symbols should be clearly defined either in the caption or in a legend provided as part of the figure. Figures should be placed at the top or bottom of a page wherever possible, as close as possible to the first reference to them in the paper. Please ensure that all the figures are of 300 DPI resolutions as this will facilitate good output.
The figure number and caption should be typed below the illustration in 8 pt and left justified [Note: one-line captions of length less than column width (or full typesetting width or oblong) centered]. For more guidelines and information to help you submit high quality artwork please visit: \url{http://www.elsevier.com/artworkinstructions} Artwork has no text along the side of it in the main body of the text. However, if two images fit next to each other, these may be placed next to each other to save space. For example, see Fig. 1. 

\section*{Acknowledgements} Acknowledgements and Reference heading should be left justified, bold, with the first letter capitalized but have no numbers. Text below continues as normal.

\section*{A. An example appendix} Authors including an appendix section should do so before References section. Multiple appendices should all have headings in the style used above. They will automatically be ordered A, B, C etc.

\subsection*{1. Example of a sub-heading within an appendix.} There is also the option to include a subheading within the Appendix if you wish.

\section*{References} References should be formatted in a numbered reference style, and listed in the order in which they appeared in the artciel text~\cite{texbook}.

\noindent N.B. Please include DOIs for all journal articles.

\bibliographystyle{plain}
\bibliography{refs.bib}

\end{document}


